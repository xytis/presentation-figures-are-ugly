\documentclass{beamer}

\usepackage[utf8]{inputenc}
\usepackage{default}
\usepackage{xltxtra}
\usepackage{fontspec}
\usepackage{caption}
\usepackage{color}
\usepackage{fontspec}
\usepackage{textcomp}

\usepackage[lithuanian]{babel}


% create \currentname command.
\usepackage{nameref}
\makeatletter
\newcommand*{\currentname}{\@currentlabelname}
\makeatother

\usefonttheme{professionalfonts}

\usecolortheme{lily}

%remove navigation symbols
\setbeamertemplate{navigation symbols}{}

\usetheme{default}

\definecolor{roasted_carrot}{RGB}{244, 130, 20}
\definecolor{text_grey}{RGB}{92, 92, 92}
\setbeamercolor{structure}{fg=roasted_carrot}
\setbeamercolor{normal text}{fg=text_grey}

\title{Gražūs grafikai}
\author[Vytis]{Vytis Valentinavičius}

\date{\scriptsize {\sc all wrongs reserved} \textcopyleft\ {\sc 2013}}

\begin{document}

\frame{\titlepage}

\frame{\tableofcontents[pausesections, hideallsubsections]}

\section{Kas yra grafikas?}
\frame{\tableofcontents[currentsection,  subsectionstyle=show/shaded/hide]}

\begin{frame}{\currentname}
  \begin{figure}
    \begin{center} 
      \includegraphics[width=0.95\textwidth]{img/graph_00.png}
      \caption*{Autobuso 5G išvykimo iš Universiteto stotelės link Pašilaičių grafikas}
    \end{center}
  \end{figure}
\end{frame}

\begin{frame}{\currentname}
  \begin{figure}
    \begin{center} 
      \includegraphics[]{img/graph_02}
      \caption*{Tranzistoriaus amplitudės ir fazės atsakas}
    \end{center}
  \end{figure}
\end{frame}

\section{Kam skirtas grafikas?}
\frame{\tableofcontents[currentsection, subsectionstyle=show/shaded/hide]}

\begin{frame}{\currentname}
  \begin{itemize}
   \only<1>{
   \item Vaizdžiai perteikti teorinę priklausomybę
    \begin{figure}
     \begin{center} 
      \includegraphics[]{img/graph_03}
     \end{center}    
    \end{figure}
   }
   \only<2>{
    \item Pademonstruoti kiekybinius tyrimo rezultatus
    \begin{figure}
     \begin{center}
      \includegraphics[]{img/graph_04}
     \end{center}    
    \end{figure}
}
  \end{itemize}
\end{frame}

\section{Kaip daromas grafikas?}
\frame{\tableofcontents[currentsection, subsectionstyle=show/shaded/hide]}

\begin{frame}{\currentname}
Bad example?
\end{frame}

\subsection{Idėja}
\frame{\tableofcontents[currentsection, subsectionstyle=show/shaded/hide]}

\begin{frame}{\currentname}
  \begin{itemize}[<+->]
    \item Žinokite, ką norite parodyti
    \only<2-5>{ 
    \hspace{0.5cm}      %Keep the first minipage below title.
    \begin{minipage}{0.48\textwidth}
      \begin{itemize}
        \item<2-> Dėsningumą\only<2>{:}
        \item<3-> Pasiskirstymą\only<3>{:}
        \item<4-> Evoliuciją\only<4>{:}
        \item<5-> Bruožus\only<5>{:}
      \end{itemize}
    \end{minipage}
    \begin{minipage}{0.41\textwidth}
      \only<2>{ \includegraphics[]{img/graph_05_pattern} }
      \only<3>{ \includegraphics[]{img/graph_05_distribution} }
      \only<4>{ \includegraphics[]{img/graph_05_evolution} }
      \only<5>{ \includegraphics[]{img/graph_05_features} }
    \end{minipage}
    }
    \only<6-8>{
    \item<6-8> Grafikas:
    \begin{itemize}
      \item<6-> turi perteikti norimą informaciją
      \item<7-> neturi būti perkrautas
      \item<8-> turi būti naudingas
    \end{itemize}
    }
    \item<9-> Pasirinkite tinkamą formą
    \only<10>{
      \begin{figure}
        \raisebox{-.5\height}{\includegraphics[]{img/graph_05_bad}}%
        \raisebox{-.5\height}{\textcolor{roasted_carrot}{$\blacktriangleright$}}%
        \raisebox{-.5\height}{\includegraphics[]{img/graph_05_good}}
      \end{figure}
    }
  \end{itemize}
\end{frame}

\subsection{Duomenys}
\frame{\tableofcontents[currentsection, subsectionstyle=show/shaded/hide]}

\subsection{Ašys}
\frame{\tableofcontents[currentsection, subsectionstyle=show/shaded/hide]}

\subsection{Tinklelis}
\frame{\tableofcontents[currentsection, subsectionstyle=show/shaded/hide]}

\subsection{Legenda}
\frame{\tableofcontents[currentsection, subsectionstyle=show/shaded/hide]}

\subsection{Antraštė}
\frame{\tableofcontents[currentsection, subsectionstyle=show/shaded/hide]}

\section{The end(???)}
\frame{\tableofcontents[currentsection, subsectionstyle=show/shaded/hide]}

\begin{frame}
  \begin{figure}
    \begin{center} 
      \includegraphics{img/ugly-figures-cropped.png}
    \end{center}
  \end{figure}
\end{frame}

\end{document}
