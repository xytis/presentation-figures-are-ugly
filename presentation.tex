\documentclass{beamer}

\usepackage[utf8]{inputenc}
\usepackage{default}
\usepackage{xltxtra}
\usepackage{fontspec}
\usepackage{caption}
\usepackage{color}
\usepackage{fontspec}
\usepackage{textcomp}
\usepackage{gensymb}

\usepackage[lithuanian]{babel}


% create \currentname command.
\usepackage{nameref}
\makeatletter
\newcommand*{\currentname}{\@currentlabelname}
\makeatother

\usefonttheme{professionalfonts}

\usecolortheme{lily}

%remove navigation symbols
\setbeamertemplate{navigation symbols}{}

\usetheme{default}

\definecolor{roasted_carrot}{RGB}{244, 130, 20}
\definecolor{text_grey}{RGB}{92, 92, 92}
\setbeamercolor{structure}{fg=roasted_carrot}
\setbeamercolor{normal text}{fg=text_grey}

\title{Gražūs grafikai}
\author[Vytis]{Vytis Valentinavičius}

\date{\scriptsize {\sc all wrongs reserved} \textcopyleft\ {\sc 2013}}

\begin{document}

\begin{frame}
  \begin{figure}
    \begin{center} 
      \includegraphics[width=0.98\textwidth]{img/convincing.png}
    \end{center}
  \end{figure}
\end{frame}

\frame{\titlepage}

\frame{\tableofcontents[pausesections, hideallsubsections]}

\section{Kas yra grafikas?}
\frame{\tableofcontents[currentsection,  subsectionstyle=show/shaded/hide]}

\begin{frame}{\currentname}
  \begin{figure}
    \begin{center} 
      \includegraphics[width=0.95\textwidth]{img/graph_00.png}
      \caption*{Autobuso 5G išvykimo iš Universiteto stotelės link Pašilaičių grafikas}
    \end{center}
  \end{figure}
\end{frame}

\begin{frame}{\currentname}
  \begin{figure}
    \begin{center} 
      \includegraphics[]{img/graph_02}
      \caption*{Tranzistoriaus amplitudės ir fazės atsakas}
    \end{center}
  \end{figure}
\end{frame}

\section{Kam skirtas grafikas?}
\frame{\tableofcontents[currentsection, subsectionstyle=show/shaded/hide]}

\begin{frame}{\currentname}
  \begin{itemize}      \only<2> {

      }
   \only<1>{
   \item Vaizdžiai perteikti teorinę priklausomybę
    \begin{figure}
     \begin{center} 
      \includegraphics[]{img/graph_03}
     \end{center}    
    \end{figure}
   }
   \only<2>{
    \item Pademonstruoti kiekybinius tyrimo rezultatus
    \begin{figure}
     \begin{center}
      \includegraphics[]{img/graph_04}
     \end{center}    
    \end{figure}
}
  \end{itemize}
\end{frame}

\section{Kaip daromas grafikas?}
\frame{\tableofcontents[currentsection, subsectionstyle=show/shaded/hide]}

\begin{frame}{\currentname}
\begin{center}
\Huge \uncover<2->{Pieštukas} \only<1-3>{+ \uncover<3>{Liniuotė}} \only<4>{=]}
\end{center}
\end{frame}

\subsection{Idėja}
\frame{\tableofcontents[currentsection, subsectionstyle=show/shaded/hide]}

\begin{frame}{\currentname}
  \begin{itemize}[<+->]
    \item Žinokite, ką norite parodyti
    \only<2-5>{ 
    \begin{flushleft}      %Keep the first minipage below title.
    \begin{minipage}{0.48\textwidth}
      \begin{itemize}
        \item<2-> Dėsningumą\only<2>{:}
        \item<3-> Pasiskirstymą\only<3>{:}
        \item<4-> Evoliuciją\only<4>{:}
        \item<5-> Bruožus\only<5>{:}
      \end{itemize}
    \end{minipage}
    \begin{minipage}{0.41\textwidth}
      \only<2>{ \includegraphics[]{img/graph_05_pattern} }
      \only<3>{ \includegraphics[]{img/graph_05_distribution} }
      \only<4>{ \includegraphics[]{img/graph_05_evolution} }
      \only<5>{ \includegraphics[]{img/graph_05_features} }
    \end{minipage}
    \end{flushleft}
    }
    \only<6->{
    \item<6-> Grafikas:
    \begin{itemize}
      \item<6-> turi perteikti norimą informaciją
      \item<7-> neturi būti perkrautas
      \item<8-> turi būti naudingas
    \end{itemize}
    }
    \item<9-> Pasirinkite tinkamą formą
    \only<10>{
      \begin{figure}
        \raisebox{-.5\height}{\includegraphics[]{img/graph_05_bad}}%
        \raisebox{-.5\height}{\textcolor{roasted_carrot}{$\blacktriangleright$}}%
        \raisebox{-.5\height}{\includegraphics[]{img/graph_05_good}}
      \end{figure}
    }
  \end{itemize}
\end{frame}

\subsection{Duomenys}
\frame{\tableofcontents[currentsection, subsectionstyle=show/shaded/hide]}

\begin{frame}{\currentname}
 \begin{itemize}[<+->]
    \item<1-> Pasirinkite teisingus matavimo vienetus
    \only<2-4>{
      \begin{itemize}
        \item<2-> Labiausiai tikėtinus
        \item<3-> Normalizuotus
      \end{itemize}
    }
    \only<4>{
      \begin{figure}
        \raisebox{-.5\height}{\includegraphics[]{img/graph_07_bad}}%
        \raisebox{-.5\height}{\textcolor{roasted_carrot}{$\blacktriangleright$}}%
        \raisebox{-.5\height}{\includegraphics[]{img/graph_07_good}}
      \end{figure}
    }
  \item<5-> Atvaizduokite tik svarbias paklaidas
    \only<6>{
      \begin{figure}
        \raisebox{-.5\height}{\includegraphics[]{img/graph_06_no_whiskers}}%
        \raisebox{-.5\height}{\textcolor{roasted_carrot}{$\blacktriangleleft\blacktriangleright$}}%
        \raisebox{-.5\height}{\includegraphics[]{img/graph_06_whiskers}}
      \end{figure}
    }
  \item<7-> Išmeskite nereikalingas vertes
    \only<8>{
      \begin{figure}
        \raisebox{-.5\height}{\includegraphics[]{img/graph_08_bad}}%
        \raisebox{-.5\height}{\textcolor{roasted_carrot}{$\blacktriangleright$}}%
        \raisebox{-.5\height}{\includegraphics[]{img/graph_08_good}}
      \end{figure}
    }
  \item<9-> Nepamirškite teorijos
  \only<9->{
    \begin{itemize}
      \item<10-> Palyginikite su teorine kreive \note{fit'inimas}
      \item<11-> Pažymėkite svarbias dalis \note{smailės, lūžiai, lygumos}
      \item<12-> Sunormuokite...
    \end{itemize}
  }
  \end{itemize}
\end{frame}


\subsection{Ašys}
\frame{\tableofcontents[currentsection, subsectionstyle=show/shaded/hide]}

\begin{frame}{\currentname}
  \begin{itemize}
    \item<1-> Skalė
    \only<2-3>{ 
    \begin{flushleft}      %Keep the first minipage below title.
    \begin{minipage}{0.48\textwidth}
      \begin{itemize}
        \item<2> Tiesinė
        \item<3> Logaritminė
      \end{itemize}
    \end{minipage}
    \begin{minipage}{0.41\textwidth}
      \only<2>{ 
        \begin{figure}[h]
          \centering
          \begin{tabular}{c}
            \includegraphics[]{img/graph_09_line_line} \\
            \includegraphics[]{img/graph_09_line_exp}
          \end{tabular}
        \end{figure}
      }
      \only<3>{ 
        \begin{figure}[h]
          \centering
          \begin{tabular}{c}
            \includegraphics[]{img/graph_09_log_line} \\
            \includegraphics[]{img/graph_09_log_exp} 
          \end{tabular}
        \end{figure} 
      }
    \end{minipage}
    \end{flushleft}
    }
    \item<4-> Ribos
    \only<4-7>{ 
      \begin{itemize}
        \item<5-> Turi apimti visą grafiką
        \item<6-> Turi apimti svarbias reikšmes \note{ 0, soties srovė ir pan. }
        \item<7-> Turi neapgaudinėti skaitytojų
      \end{itemize}
    }
    \item<8-> Kiekybė
    \only<9>{ 
      \begin{figure}[h]
        \includegraphics[]{img/graph_06_crowded}
      \end{figure} 
    }
  \end{itemize}
\end{frame}

\begin{frame}{\textcolor{text_grey}{HELL}NO}
  \begin{figure}[h]
    \includegraphics[height=0.8\textheight]{img/hellno}
  \end{figure} 
\end{frame}


\subsection{Tinklelis}
\frame{\tableofcontents[currentsection, subsectionstyle=show/shaded/hide]}
\begin{frame}{\currentname}
  \begin{itemize}[<+->]
    \item Ar pavieniai duomenys yra svarbūs?
    \item Ar grafikas yra komplikuotas?
    \item Ar Jums laborai pas P.J. Žilinską?
  \end{itemize}
\end{frame}

\begin{frame}{\currentname}
  \begin{figure}
    \begin{center} 
      \includegraphics[]{img/graph_02}
      \caption*{Tranzistoriaus amplitudės ir fazės atsakas}
    \end{center}
  \end{figure}
\end{frame}

\subsection{Legenda}
\frame{\tableofcontents[currentsection, subsectionstyle=show/shaded/hide]}
\begin{frame}{\currentname}
  \begin{itemize}[<+->]
    \item Ar turite daugiau negu vieną kreivę?
    \item Ar nepavyksta naudoti užrašų ties kreivėmis?
  \end{itemize}
\end{frame}

\begin{frame}{\currentname}
 \begin{figure}[h]
    \includegraphics[]{img/photo_luminescense}
  \end{figure} 
\end{frame}

\subsection{Antraštė}
\frame{\tableofcontents[currentsection, subsectionstyle=show/shaded/hide]}

\begin{frame}{\currentname}
  \begin{itemize}[<+->]
    \item<1-> Kas?
    \item<3-> ``Išmatuota judrio priklausomybė nuo temperatūros''
    \item<2-> Kodėl?
    \item<4-> ``n-Ge judris ties $T=179\degree$ K pasiekia maksimalią vertę $\mu=237$ cm/s''
    \item<5-> Trumpa.
  \end{itemize}
\end{frame}

\frame{\tableofcontents[sections={3}]}

%\section{The End}
\begin{frame}[plain,c]
\begin{center}
\Huge The End
\end{center}
\end{frame}

\begin{frame}
  \begin{figure}
    \begin{center} 
      \includegraphics{img/ugly-figures-cropped.png}
    \end{center}
  \end{figure}
\end{frame}

\end{document}
